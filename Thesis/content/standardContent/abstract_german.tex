\chapter*{Kurzreferat}

\subsection*{Integration eines Feature-orientierten Testsystems in den Entwicklungszyklus technischer Systeme}

Die Entwicklung technischer Systeme ist ein komplexer und kostspieliger Prozess. Daher ist es wichtig, dass die Produkte vor der Auslieferung an den Kunden gezielt und sorgfältig getestet werden. Dies wird durch Zeitdruck und Deadlines oftmals vernachlässigt, oder nur unzureichend durchgeführt. Mit einem gut strukturierten Testplan kann dieses Risiko allerdings minimiert werden, und die Durchführung der Tests mit dem Produktentwicklungszyklus integriert werden. Dadurch kann stabile Hardware sowie effiziente und gut strukturierte Software ohne große Verzögerung an den Kunden ausgeliefert werden.

\smallskip

Durch Methodiken wie zum Beispiel Feature-orientierte Entwicklung ist es möglich, dass bestimmte Features vor dem Release der Produkte abgenommen und getesten werden müssen. Das wiederum ermöglicht, dass neu implementierte Features gründlich getestet werden und somit zu einer hohen Qualität beitragen.

\smallskip

Dies kann erreicht werden, indem Features strukturiert in einer Datenbank eingepflegt werden. Ein auf diesen Featurebeschreibungen basierendes System kann die automatische Testdurchführung unterstützen, gezielt Tests für einzelne Features durchführen und deren Abnahme beschleunigen. Dadurch kann der Testaufwand verringert und den Entwicklern ein fokussiertes Feedback vermittelt werden.

\clearpage