% !TeX document-id = {f5eab2f7-1326-43b3-ac0e-350c2acbded8}

% !TeX TXS-program:compile = txs:///pdflatex/[--shell-escape]

\documentclass[a4paper, fontsize=11pt, parskip=half, twoside]{scrreprt}

\usepackage[utf8]{inputenc}
\usepackage[T1]{fontenc}   
\usepackage{graphicx}       
\usepackage[english, ngerman]{babel}
% for displaying quotes
\usepackage{csquotes}     
\usepackage{acronym}
\usepackage{eurosym}
\usepackage[linktocpage=true]{hyperref}
\usepackage{xurl}
\usepackage[bindingoffset=8mm]{geometry}
\usepackage{caption}
\captionsetup{format=hang, justification=raggedright}
\usepackage[style=authoryear, backend=biber]{biblatex}
\usepackage{float}
\usepackage{rotating}
\usepackage{amsmath}
\usepackage{amssymb}

% for drawing
\usepackage{tikz}

% improve micro typography for better distribution
\usepackage{microtype}

% adds generic commands for degree, ohm, ..
\usepackage{textcomp}
\usepackage{gensymb}

% for source code
\usepackage{minted}

% caption for e.g. figures
\usepackage{caption}
% captions for multiple figures
\usepackage{subcaption}
% for custom dates
\usepackage{datetime}
% clever refs
% set labels as always (\label{fig:hello}), \cref{fig:hello}
\usepackage[german]{cleveref}

% 1.5x line space
\usepackage[onehalfspacing]{setspace}

\usepackage{todonotes}

% define date variable for whole doc
% use \displaydate{date} where needed to insert this date

\newdate{date}{01}{07}{2023}
\date{\displaydate{date}}

% add zotero file for citations
\addbibresource{reference.bib}

\begin{document}
	
	\thispagestyle{empty}
	
	\cleardoublepage   % force output to a right page
	\thispagestyle{empty}
	\begin{titlepage}
		\begin{flushright}
			\includegraphics[width=0.4\linewidth]{assets/Logo-A3.jpg}
		\end{flushright}
		
		\begin{flushleft}
			\section*{Integration eines Feature-orientierten Testsystems in den Entwicklungszyklus technischer Systeme}
			%\subsection*{\papersubtitle}
			\vspace{1cm}
			
			\vspace{0.5cm}
			Bachelorarbeit\newline
			zum Erlangen des akademischen Grades\newline
			
			\vspace{0.5cm}
			\textbf{Bachelor of Science in Engineering (BSc)}
			
			\vspace{1cm}
			Fachhochschule Vorarlberg\newline
			Informatik - Software and Information Engineering\newline
			
			\vspace{0.5cm}
			Betreut von\newline
			Dipl.-Ing. Dr. techn. Ralph Hoch
			
			\vspace{0.5cm} 
			Vorgelegt von\newline
			Marco Prescher	
			
			\vspace{0.5cm}
			Dornbirn, am \displaydate{date}
		\end{flushleft}
	\end{titlepage}
	
	% Kurzreferat:
	\newpage
	\section*{Kurzreferat}
	\subsection*{Integration eines Feature-orientierten Testsystems in den Entwicklungszyklus technischer Systeme}
	
	Die Entwicklung technischer Systeme ist ein komplexer und kostspieliger Prozess. 
	Daher ist es wichtig, dass die Produkte vor der Auslieferung an den Kunden gezielt und sorgfältig getestet werden. 
	Dies wird durch Zeitdruck und Deadlines oftmals vernachlässigt, oder nur unzureichend durchgeführt. 
	Mit einem gut strukturierten Testplan kann dieses Risiko allerdings minimiert werden, und die Durchführung der Tests mit dem Produktentwicklungszyklus integriert werden. 
	Dadurch kann stabile Hardware sowie effiziente und gut strukturierte Software ohne große Verzögerung an den Kunden ausgeliefert werden.
	
	Durch Methodiken wie zum Beispiel Feature-orientierte Entwicklung ist es möglich, dass bestimmte Features vor dem Release der Produkte abgenommen und getestet werden müssen. 
	Das wiederum ermöglicht, dass neu implementierte Features gründlich getestet werden und somit zu einer hohen Qualität beitragen.
	
	Dies kann erreicht werden, indem Features strukturiert in einer Datenbank eingepflegt werden. 
	Ein auf diesen Featurebeschreibungen basierendes System kann die automatische Testdurchführung unterstützen, gezielt Tests für einzelne Features durchführen und deren Abnahme beschleunigen. 
	Dadurch kann der Testaufwand verringert und den Entwicklern ein fokussiertes Feedback vermittelt werden.
	
	\newpage
	\section*{Abstract}
	\subsection*{Integrating a feature-oriented testing system into the development cycle of technical systems}
	
	The development of technical systems is a complex and costly process. 
	Therefore, it is important that products are thoroughly tested before delivery to the customer. 
	However, this is often neglected or done insufficiently due to time pressure and deadlines. 
	A well-structured test plan can minimize this risk and integrate the testing process into the product development cycle. 
	This allows stable hardware and efficient, well-structured software to be delivered to the customer without significant delays.
	
	Using techniques such as feature-oriented development, certain features must be accepted and tested before the product release. 
	This in turn allows newly implemented features to be thoroughly tested and contribute to high quality. 
	
	This can be achieved by structuring features in a database. 
	A system based on these feature descriptions can support automatic testing, conduct targeted tests for individual features, and accelerate their acceptance. 
	This reduces testing effort and provides focused feedback to the developers.
	
	% Inhaltsverzeichnis:
	\cleardoublepage   % force output to a right page
	\setcounter{tocdepth}{2}
	\setcounter{secnumdepth}{4}
	\tableofcontents
	
	% Abbildungsverzeichnis
	\clearpage
	\phantomsection
	\addcontentsline{toc}{chapter}{Abbildungsverzeichnis}
	\listoffigures
	
	% Abkürzungsverzeichnis:
	\clearpage
	\phantomsection
	\addcontentsline{toc}{chapter}{Abkürzungsverzeichnis}
	\section*{Abkürzungsverzeichnis}
	\begin{acronym}
		\acro{API}{Application Programming Interface}
		\acro{JSON}{JavaScript Object Notation}
		\acro{SPA}{Single Page Application}
		\acro{TCMS}{Test Case Management System}
		\acro{TDD}{Test Driven Development}
		
		\acrodefplural{TCMS}{Test Case Management Systeme}
	\end{acronym}
	
	\clearpage
	\section*{Danksagung}
	Ich möchte mich aufrichtig bei Ralph Hoch, der Firma Gantner Instruments und allen Mitarbeitenden bedanken, die mir bei der Vollendung dieser Arbeit geholfen haben. 
	Ihre Unterstützung und Expertise waren von unschätzbarem Wert und haben zum erfolgreichen Abschluss dieses Projekts beigetragen. 
	Vielen Dank für Ihre harte Arbeit und Ihr Engagement. 
	Ich schätze Ihre Zusammenarbeit sehr.
	
	
	\clearpage
	\chapter{Einleitung}
	\todo{Einleitung}
	
	\ac{TDD}
	\ac{TDD}
	\textcite{harman_well-wrought_2012}
	\cref{fig:schelling_end}
	
	\cite{harman_well-wrought_2012}
	
	\section{Motivation}
	Die Entwicklung technischer Systeme ist ein komplexer Prozess, der eine hohe Qualität erfordert, um den Anforderungen der Kunden gerecht zu werden. 
	Damit diese Qualität auch gewährleistet wird, müssen Fehler sowie Mängel identifiziert und ausgebessert werden. 
	Ein \ac{TCMS} bietet eine Lösung, um diesen Prozess zu vereinfachen und effektiver zu gestalten. 
	Durch die Verwendung eines \ac{TCMS} können Angestellte, die in der Software/Hardware Entwicklung, Support, Marketing etc., die Qualität des Produkts verbessern, Fehler frühzeitig erkennen, beheben und den Entwicklungsprozess effizienter gestalten. 
	Diese Arbeit fokussiert sich daher auf die Integration von einem \ac{TCMS} in einen laufenden Produktentwicklungs-Zyklus und vergleicht verschiedene vorhandene Lösungen.
	
	\section{Problemstellung}
	Die Komplexität von Projekten steigt kontinuierlich an und es ist eine Herausforderung, die Qualität des entwickelten Systems sicherzustellen. 
	Es ist bekannt, dass Fehler, die erst spät im Entwicklungsprozess entdeckt werden, viel kostspieliger zu beheben sind als Fehler, die frühzeitig identifiziert und behoben werden. 
	Infolgedessen suchen Unternehmen nach Lösungen, um den Testprozess effektiver zu gestalten und Fehler früher im Entwicklungszyklus zu identifizieren. 
	\ac{TCMS} bietet eine solche Lösung, indem es Entwicklern und Testern ermöglicht, Testfälle effizient zu planen und zu verwalten sowie Testergebnisse zu erfassen, darzustellen und somit auch zu verfolgen. 
	Obwohl \aclp{TCMS} in der Industrie weit verbreitet sind und es einige fertige Lösungen gibt, gibt es jedoch nicht immer die Perfekte Lösung um ein bestehendes \ac{TCMS} für das eigene Projekt anzuwenden. 
	Insbesondere gibt es Bedenken hinsichtlich der Anwendbarkeit von \ac{TCMS} mit anderen Tools und der Skalierbarkeit von \ac{TCMS}. 
	Diese Probleme stellen Hindernisse dar, die die Einführung von \ac{TCMS} in einem Unternehmen erschweren können. 
	In dieser Arbeit werden wir uns mit diesen Problemen und Herausforderungen auseinandersetzen, eine Software Lösung entwickeln und untersuchen, wie \aclp{TCMS} effektiv eingesetzt werden können, um den Testprozess zu optimieren und die Qualität eines Produkts zu verbessern.
	
	\section{Zielsetzung}
	Das Ziel dieser Arbeit ist es, die Verwendung von \ac{TCMS}en zu untersuchen und zu bewerten. 
	Zudem ein auf unser eigenes Produkt angepasstes \ac{TCMS} backend zu entwickeln. 
	Dabei heraus ergeben sich drei relevante fragen:
	
	\begin{itemize}
		\item Wie kann man Features beschreiben? 
		\item wie in Datenbank schreiben? 
		\item und wie von da aus bearbeiten?
	\end{itemize}
	
	Insbesondere möchten wir die folgenden Ziele erreichen:
	
	\begin{itemize}
		\item Die Vor- und Nachteile der Verwendung von \ac{TCMS} zu identifizieren und zu analysieren.
		\item Die Entwicklung und Integration von \ac{TCMS} in den Entwicklungsprozess zu untersuchen und zu bewerten, einschließlich der Herausforderungen, die bei der Integration von \ac{TCMS} in bestehende Entwicklungs- und Testprozesse auftreten können.
		\item Die Konnektivität von \ac{TCMS} mit anderen Tools und Systemen, die in der Softwareentwicklung verwendet werden, zu untersuchen und zu bewerten.
		\item Empfehlungen für die erfolgreiche Implementierung von \ac{TCMS} in der Softwareentwicklung zu geben, einschließlich der Identifizierung bewährter Praktiken.
	\end{itemize}
	
	Durch die Erfüllung dieser Ziele wird diese Arbeit dazu beitragen, das Verständnis für die Verwendung von \ac{TCMS} in der Produktentwicklung zu verbessern und Unternehmen dabei zu unterstützen, den Testprozess zu optimieren und die Qualität ihrer Produkte zu verbessern.
	
	\chapter{Stand des Wissens} \label{sec:stateofart}
	In diesem Kapitel wird eine systematische Überprüfung der angewandten Technologien vorgenommen. 
	Dabei werden die verschiedenen Typen von Tests, die Verwendung von Testmanagement-Systemen sowie die Organisation und Strukturierung von Funktionalitäten gründlich analysiert.
	
	
	
	\section{Testarten}
	\todo{Proper Testing Description}
	
	\subsection{Unit Test}
	\todo{Proper Unit Test Description}
	
	\subsection{Integration Test}
	\todo{Proper Integration Test Description}
	
	
	\section{Einführung in das Test Case Management}
	
	Ein \ac{TCMS} ist eine Software, die verwendet wird, um Testfälle für ein bestimmtes Projekt oder eine Anwendung zu verwalten und zu organisieren. 
	Es hilft bei der Planung, Überwachung und Dokumentation von Tests und ermöglicht es, Testfälle sicher und effizient zu verwalten.
	
	Ein \ac{TCMS} verfügt über Funktionen wie Testfall-Erstellung, Testfall-Verwaltung, Testfall-Ausführung und Ergebnisberichterstattung. 
	Es kann auch eine integrierte Umgebung für die Zusammenarbeit von Testern und Entwicklern bereitstellen.
	
	Zusammenfassend ist \ac{TCMS} ein wichtiges Werkzeug, um einen strukturierten und effektiven Testprozess zu gewährleisten und den Qualitätsstandard einer Anwendung zu verbessern.
	
	\subsection{Überblick über Test Case Management Systeme}
	\todo{Proper Überblick über Test Case Management Systeme Description}
	
	\subsection{Funktionen und Eigenschaften von Test Case Management Systemen}
	\todo{Proper Funktionen und Eigenschaften von Test Case Management Systemen Description}
	
	\subsection{Methoden und Techniken im Test Case Management}
	\todo{Proper Methoden und Techniken im Test Case Management Description}
	
	\subsection{Aktuelle Trends und Herausforderungen}
	\todo{Proper Aktuelle Trends und Herausforderungen Description}
	
	\subsection{Bewertung von Test Case Management Systemen}
	\todo{Proper Bewertung von Test Case Management Systemen Description}
	
	\section{Relationale Datenbank}
	\todo{Proper Relationale Datenbank Description Description}
	
	\subsection{MariaDB}
	\todo{Proper MariaDB Description}
	
	
	\section{Docker}
	\todo{Proper Docker Description}
	
	
	
	\chapter{Lösungsansatz}
	\todo{Proper Lösungsansatz}
	
	
	\section{Scrum}
	\todo{Proper Scrum Description}
	
	
	\section{Architektur}
	\todo{Proper Architektur Description}
	
	\subsection{Backend}
	\todo{Proper Backend Description}
	
	
	\section{Vorentwicklungsphase}
	\todo{Proper Vorentwicklungsphase Description}
	
	\subsection{Projektstruktur}
	\todo{Proper Projektstruktur Description}
	
	\subsection{Domain Model}
	\todo{Proper Domain Model Description}
	
	\subsection{ER Model}
	\todo{Proper ER Model Description}
	
	\subsection{Featurestruktur}
	\todo{Proper Featurestruktur Description}
	
	\subsection{Featurebedingungen}
	\todo{Proper Featurebedingungen Description}
	
	\subsection{Featureverwaltung}
	\todo{Proper Featureverwaltung Description}
	
	
	\section{Entwicklungsphase}
	\todo{Proper Entwicklungsphase Description}
	
	\subsection{Verwaltung und Struktur von Features}
	\todo{Proper Verwaltung und Struktur von Features Description}
	
	\subsection{Featurestruktur zur Datenbank}
	\todo{Proper Featurestruktur zur Datenbank Description}
	
	\subsection{Visualisierung von Featurestrukturen}
	\todo{Proper Visualisierung von Featurestrukturen Description}
	
	
	
	\chapter{Implementierung}
	\todo{Proper Implementierung}
	
	
	\section{Sprints}
	\todo{Proper Sprints Description}
	
	\section{User Stories}
	\todo{Proper User Stories Description}
	
	\subsection{Story 1}
	\todo{Proper Story 1 Description}
	
	\subsection{Story 2}
	\todo{Proper Story 2 Description}
	
	\subsection{Story n}
	\todo{Proper Story n Description}
	
	
	
	\chapter{Ergebnisse}
	\todo{Proper Ergebnisse Description}
	
	
	
	\chapter{Fazit und Ausblick}
	\todo{Proper Fazit und Ausblick}
	
	
	
	\begin{figure}[ht]
		\begin{minted}[autogobble, tabsize=4]{rust}
			{
				// Start eines neuen Scopes
				
				let hello = String::from("hello");
				// Ende des Scopes. hello wird an dieser Stelle freigegeben
			}
			// hello kann hier nicht mehr verwendet werden
		\end{minted}
		\caption{Ein String wird in einem Scope erzeugt und der Variable hello zugewiesen. Am Ende des Scopes wird der Speicher freigegeben.}
		\label{fig:drop:example}
	\end{figure}
	
	\begin{figure}[H]
		\centering
		\includegraphics[width = 10cm]{assets/Logo-A3.jpg}
		\caption{Nach 100 Simulationsschritten, einem minimalen Gruppenanteil $B{\min} = 0.4$ und einer Nachbarschaftsgröße von einem Feld sind die Gruppen bereits sichtlich separiert.}
		\label{fig:schelling_end}
	\end{figure}
	
	% Literaturverzeichnis:
	\clearpage
	\phantomsection
	\addcontentsline{toc}{chapter}{Literaturverzeichnis}
	\printbibliography
	
	\clearpage
	\section*{Eidesstattliche Erklärung}
	Ich erkläre hiermit an Eides statt, dass ich vorliegende Bachelorarbeit selbstständig und ohne Benutzung anderer als der angegebenen Hilfsmittel angefertigt habe. 
	Die aus fremden Quellen direkt oder indirekt übernommenen Stellen sind als solche kenntlich gemacht. 
	Die Arbeit wurde bisher weder in gleicher noch in ähnlicher Form einer anderen Prüfungsbehörde vorgelegt und auch noch nicht veröffentlicht.
	
	\vspace{3cm}
	\noindent
	Dornbirn, am \displaydate{date} \hfill Marco Prescher
	
	
\end{document}