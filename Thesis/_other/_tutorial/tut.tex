\chapter{Überblick Latex}\label{ch:overview-latex}
\section{Acronyms}
    First occurrence: \ac{FDI} and \ac{MNE}\\
    Second occurrence: \ac{FDI}\\
    Force full: \acf{FDI}
\section{Cite}
    Cite normal: \cite{HRCS} \\
    Cite test: \cite{george_arnett_foreign_2015} \\
    Cite with parenthesis: \citep{george_arnett_foreign_2014} \\
    Cite Author: \citeauthor{george_arnett_foreign_2014} \\
    Cite with additional info: \citet[Page 2, Fig. I.1]{george_arnett_foreign_2014} \\
    Cite with parenthesis with additional info: \citep[Page 2, Fig. I.1]{george_arnett_foreign_2014}

\section{Images}
\begin{figure}[H] 
  \centering
    \includegraphics[angle=90]{images/Logo-A3.jpg}%
  \caption{Rotated by 90 degrees}
  \label{fig:rotatedImg}
\end{figure}
\begin{figure}[H] 
  \centering
    \includegraphics[width=1\textwidth,angle=0,scale=1]{images/Logo-A3.jpg}%
  \caption{Normal bound to width of page}
  \label{fig:em202y}
\end{figure}
    
\section{Code}
\subsection{Multiline}
\begin{longlisting}
\begin{minted}{js}
export const calculateUserTime = (timestamp) => {
    if (timestamp) {
        if (timestamp.includes && timestamp.includes('T') && timestamp.includes('Z')) {
            return DateTime.fromISO(timestamp).setZone(useTimezone)
            .toLocaleString(DATE_FORMAT); //ISO format
        } else {
            return getDateFromFormat(timestamp).toLocaleString(DATE_FORMAT); 
            //System time -> string format
        }
    } else {
        return DateTime.now().setZone(useTimezone).toLocaleString(DATE_FORMAT); 
        //setZone -> to convert utc date into selected timezone
    }
};
\end{minted}
\caption{Beispiel Benutzer Zeit berechnen}
\label{lst:jetpack-compose-example}
\end{longlisting}

\subsection{Inline}
Text mit code \mint{html}|<h2>Something <b>here</b></h2>| inlined

\section{Tables}
Tables:
\url{https://www.tablesgenerator.com/}
\begin{table}[H]
\resizebox{\textwidth}{!}{%
\begin{tabular}{|l|l|l|l|}
\hline
Name              & Necessary & Definition                                         & Default                                                        \\ \hline
-src              & yes       & Source image                                       & Path: imageprocessing/src/main/resources/loetstellen.jpg       \\ \hline
-acc              & yes       & Accuracy                                           & 3                                                              \\ \hline
-pull             & yes       & Switch to Pullpipe                                 & false                                                          \\ \hline
-exptCoords       & /         & File with expected Coords                          & Path: imageprocessing/src/main/resources/expectedCentroids.txt \\ \hline
-disksImgOut      & /         & Output for created DiskImg after Erosion           & Path: imageprocessing/target/disks.png                         \\ \hline
-diskReportOut    & /         & Output for created Report                          & Path: imageprocessing/target/report.txt                        \\ \hline
-shapeTypeErode   & /         & Shape type for erode filter Range{[}0 - 3{]}       & 0                                                              \\ \hline
-kernelSizeErode  & /         & Kernel size for erode filter                       & 2                                                              \\ \hline
-kernelSizeMedian & /         & Kernel size for median filter                      & 19                                                             \\ \hline
-removeFirstN     & /         & Removes the first n found disks in the given image & 1                                                              \\ \hline
\end{tabular}%
}
\caption{Table of command-line arguments}
\label{tab:args-table}
\end{table}

\section{Lists}
\begin{itemize}
    \item[1)] Learn Mode: The user is provided with additional info to learn the ten-finger-system
    \item[2)] Standard / Progress Mode: The user is given with a series of words which they have to type within a specified time frame (e.g. 1 min)
    \begin{itemize}
        \item[a)] Words form Dictionary -> Enforce Learning with likely known words
        \item[b)] Words with random characters-> Enforce 'proper' typing with random characters because there are unambiguous words
    \end{itemize}
    \item[3)] Optional: Competitive Battle - Let Users compete against each other in a direct competition.
    \item Test
\end{itemize}

\section{Automatic Lists}

\begin{enumerate}
\item erste Ebene
\begin{enumerate}
\item zweite Ebene
\begin{enumerate}
\item dritte Ebene
\begin{enumerate}
\item vierte Ebene
\end{enumerate}
\item wieder auf dritter Ebene 
\item noch ein Eintrag 
\end{enumerate}
\item hier ist die zweite Ebene
\end{enumerate}
\item und hier die erste Ebene
\end{enumerate}

\subsection{a), b), ...}

\renewcommand{\labelenumi}{\alph{enumi})}
\begin{enumerate}
\item Eins
\item Zwei
\item Drei
\end{enumerate}

\subsection{Roman}

\renewcommand{\labelenumi}{\roman{enumi}}
\begin{enumerate}
\item Eins
\item Zwei
\item Drei
\end{enumerate}

\section{Rotate Page}
\begin{landscape}
\thispagestyle{empty}
\begin{figure}
    \centering
    \includegraphics[width=1.5\textheight]{images/Logo-A3.jpg}
    \caption{Picture on page in landscape}
    \label{fig:other}
\end{figure}
\end{landscape}